
20/09/2020
Tried blurring the level set. Technique works peaty fine depending on the kernel size and offset.
Varying kernel offset doesn't work pretty well, at least at a cell size, larger, than diameter of particle. 
Need to check it on the lower cell size (currently i am testing with a cell size equal to a particle diameter).
Artifacts: holes in the mesh.
Reason of the artifacts: the kernel offset takes neighbors too far from the blurred cell, so that its level set value is changed too much
Possible solution: blur level set depending on the curvature and thinness of the regions, e.g. in 
	high curvature regions and thin regions (where fluid is very thin) or there are two fluids near to each other - blur with lower the kernel offset

TODO: experiment on lower cell sizes

Varying kernel size gives good result of smoothing. Also solves a problem of ZhuBridson method of. Artifacts are the same as in previous. 
Some sharp features are dramatized.

Also added an option for smoothing factor, 
TODO: test reconstruction taking different smoothing factors
TODO: currently weights are taken between the level set values (ZhuBridson and blurred ZhuBridson). Need to try different Constructive Solid Geometry techniques

Cons blurring: Increased computation time due to additional blurring pass (potentially can be parallelized on GPU so that the problem is not severe)
				Artifacts in the final mesh (see abow)
24.09.2020
Global TODO's:

1. Check Zhu-Bridson what if the particle radius matches reconstructed surface. Find the problem if exists
2. Make more efficient usage of the memory for large simulations
3. Check what is the issue with the blurring technique
4. Finish the "Level set methods" theory
5. Finish article from Jose
6. Implement blurring in the direction of the gradient of scalar field
7. Think of the smoothing methods, How to apply them on the 3d grid
8. Optimize triangle mesh computation (avoid redundant points computation, apply decimation)
9. Smoothing normals of the surface for better rendering

Fixed today:
ZhuBridson reconstruction surface was 

New idea, how to enhance blurring method:
	a. Compute gradient of scalar field using forward/backward/central difference
	b. Blur w.r.t. neighbors, that are in the direction of the gradient of the blurring
	c. Pick kernel size w.r.t. the curvature, which can be determined according to the difference between the gradient direction of two grid cells

Possible titles:
	What am i doing: Find exact and efficient (gpu/cpu, memory) and highly parallelized method grid based method, for reconstruction of the smooth surface of SPH fluid

	 Exact and parallel grid based method of SPH fluid surface reconstruction
	 Reconstruction of SPH fluid surface using gpu parallelizable grid based approach

25.09.2020
TODO's:

3. Check what is the issue with the blurring technique
6. Implement blurring in the direction of the gradient of scalar field
7. Think of the smoothing methods, How to apply them on the 3d grid
4. Finish the "Level set methods" theory
5. Finish article from Jose
8. Optimize triangle mesh computation (avoid redundant points computation, apply decimation)
9. Smoothing normals of the surface for better rendering
10. Regarding the thinness problem: check out Markstein et. all.:  

Something i read and understood about convection:
	Convection is a technique of the surface propagation using the dynamic information of the fluid, like velocity. E.g. given Level set we can propagate its values between the frames given the velocity information solving the PDE: $\dfrac{dx}\dfrac{dt} = V(x)$.
	For computing implicit surface changes at $x:\phi(x) = 0$ we solve equation: $\phi_t + V * \ \phi$

20/09/2020
Tried blurring the level set. Technique works peaty fine depending on the kernel size and offset.
Varying kernel offset doesn't work pretty well, at least at a cell size, larger, than diameter of particle. 
Need to check it on the lower cell size (currently i am testing with a cell size equal to a particle diameter).
Artifacts: holes in the mesh.
Reason of the artifacts: the kernel offset takes neighbors too far from the blurred cell, so that its level set value is changed too much
Possible solution: blur level set depending on the curvature and thinness of the regions, e.g. in 
	high curvature regions and thin regions (where fluid is very thin) or there are two fluids near to each other - blur with lower the kernel offset

TODO: experiment on lower cell sizes

Varying kernel size gives good result of smoothing. Also solves a problem of ZhuBridson method of. Artifacts are the same as in previous. 
Some sharp features are dramatized.

Also added an option for smoothing factor, 
TODO: test reconstruction taking different smoothing factors
TODO: currently weights are taken between the level set values (ZhuBridson and blurred ZhuBridson). Need to try different Constructive Solid Geometry techniques

Cons blurring: Increased computation time due to additional blurring pass (potentially can be parallelized on GPU so that the problem is not severe)
				Artifacts in the final mesh (see abow)